\documentclass[12pt]{article}
%\documentclass[a4paper,11pt]{article}


%deklaracje pakietów
\usepackage{amsfonts}
\usepackage{amsmath}
\usepackage{amssymb}
\usepackage{latexsym}
\usepackage{eucal}
\usepackage{listings}

%pakiety z językiem polskim
\usepackage{polski}
\usepackage[utf8]{inputenc}
\usepackage[polish]{babel}
\usepackage[cp1250]{inputenc}
\usepackage{t1enc}
%ustawienia marginesów
\textwidth=16cm \textheight=23cm \oddsidemargin=0.5cm
\topmargin=-1cm

\usepackage{graphicx}

\renewcommand\baselinestretch{1.1}


\DeclareMathOperator{\Q}{\mathbb{Q}}
\def\R{\mathbb{R}}


\def\N{\mathbb{N}}
\def\CC{\mathcal{C}} % kaligraficzne C
\def\FF{\mathcal{F}} % kaligraficzne F
\def\MM{\mathcal{M}} % kaligraficzne M
\def\NN{\mathcal{N}} % kaligraficzne N

\newtheorem{tw}{Twierdzenie}
\newtheorem{lem}{Lemat}
\newtheorem{wn}{Wniosek}
\newtheorem{fakt}{Fakt}
\newtheorem{uw}{Uwaga}
\newtheorem{defin}{Definicja}
\newtheorem{prz}{Przykład}
\newenvironment{dow}{\par \noindent \emph{Dowód.\ }}{\par\noindent\hfill$\Box$}



\usepackage{xcolor}
\definecolor{codegreen}{rgb}{0,0.6,0}
\definecolor{codegray}{rgb}{0.5,0.5,0.5}
\definecolor{codepurple}{rgb}{0.58,0,0.82}
\definecolor{backcolour}{rgb}{0.95,0.95,0.92}
\lstdefinestyle{mystyle}{
backgroundcolor=\color{backcolour},  
commentstyle=\color{codegreen},
keywordstyle=\color{magenta},
numberstyle=\tiny\color{codegray},
stringstyle=\color{codepurple},
basicstyle=\ttfamily\footnotesize,
breakatwhitespace=false,        
breaklines=true,                
captionpos=b,                    
keepspaces=true,                
numbers=left,                    
numbersep=5pt,                  
showspaces=false,                
showstringspaces=false,
showtabs=false,                  
tabsize=2
}
\lstset{style=mystyle}
\begin{document}
 \begin{center}
\LARGE{UNIWERSYTET GDAŃSKI}
\end{center}

\begin{center}
\LARGE{WYDZIAŁ MATEMATYKI, FIZYKI I\\ INFORMATYKI}
\end{center}


\vspace{50mm}
\begin{center}
\Large{\textbf{Ewa Bojke}}\\ \vskip 4mm
\large{MODELOWANIE MATEMATYCZNE \\I ANALIZA DANYCH}
\vspace{10mm}



\Huge\textbf{Szeregi Fouriera - Maxima} 
\end{center}

\vspace{30mm}

\begin{flushright}
Projekt zaliczeniowy\\ 
dr. Marta Frankowska 
\end{flushright}
\vspace{20mm}
\begin{center}
\textbf{Gdańsk 2022}
\end{center}

\newpage
Dana jest funkcja $f(x)$ określona na przedziale $[0,t]$ taka, że

$$f(x) = 0.1x^3 - 4.5x, \\  t= 5.$$ 

\bigskip

Pierwszym etapem będzie przedłużenie funkcji $f$ do funkcji parzystej $f_p$ oraz do funkcji nieparzystej $f_n$ na przedziale $[-t,t]$ spełniającej warunki Dirichleta.

\smallskip
Sprawdźmy, czy funkcja $f(x)$ spełnia warunki Dirichleta, czyli

\begin{itemize}
    \item $f(x)$ jest przedziałami monotoniczna na odcinku $[-5,5]$,
    \item funkcja ${\displaystyle f(x)}$ w przedziale jednego okresu posiada skończoną liczbę punktów nieciągłości pierwszego rodzaju,
    \item wartości funkcji $f(x)$ na krańcach przedziału $[-5,5]$ są równe, czyli $f(-5) = f(5).$
\end{itemize}

\begin{figure}[ht!]
\centering
\includegraphics[width=100mm,scale=1]{granice.png}
\label{fig:granice}
\end{figure}

\textbf{Wnioski:} 
Funkcja $f(x)$ nie spełnia warunków Dirichleta, co widać poniżej:

\begin{figure}[ht!]
\centering
\includegraphics[width=150mm,scale=1]{warunek.png}
\label{fig:warunek}
\end{figure}


\newpage
\smallskip

Widzimy że wartości funkcji na krańcach przedziału nie są równe średniej arytmetycznej 
granic. Funkcja f nie spełnia warunków Dirichleta, musimy zatem zamienić krańcowa wartość funkcji nieparzystej na 0.

\bigskip

\bigskip

Przedłużmy funkcję $f(x)$ do funkcji parzystej $f_p$ i funkcji nieparzystej $f_n$ zgodnie ze wcześniejszymi założeniami.

\bigskip

\begin{figure}[ht!]
\centering
\includegraphics[width=160mm,scale=1]{funkcje_rozszerzenie.PNG}
\label{fig:fp-fn}
\end{figure}

\bigskip

Poniżej zamieszczony został wykres tych dwóch funkcji.

\begin{figure}[ht!]
\centering
\includegraphics[width=163mm,scale=1]{wykres_f.PNG}
\caption{Ilustracja funkcji parzystej i nieparzystej f(x)}
\label{fig:wykres2d}
\end{figure}


\newpage
Podajmy współczynniki rozwinięcia funkcji $f(x)$ w szereg Fouriera w przedziale [-5,5].

\begin{figure}[ht!]
\centering
\includegraphics[width=118mm,scale=1]{rozwiniecie-fourier.PNG}
\label{fig:fourier}
\end{figure}

\bigskip

A tak prezentują się współczynniki rozwinięcia funkcji $f(x)$ w szereg Fouriera według
sinusów i cosinusów.

\begin{figure}[ht!]
\centering
\subfloat{\label{fig:rozwiniecie-sinus}}
\includegraphics[width=125mm,scale=1]{rozwiniecie-sinus.PNG}
\quad
\subfloat{\label{fig:rozwiniecie-cosinus}}
\includegraphics[width=135mm,scale=1]{rozwiniecie-cosinus.PNG}
\caption{Rozwinięcie w szereg Fouriera wg. sinusów i cosinusów}
\end{figure}


\newpage
Dla każdego z tych szeregu zdefiniowano funkcję, która wyznacza n-tą sumę częściową:

\bigskip

\begin{figure}[ht!]
\centering
\includegraphics[width=150mm,scale=1]{suma.PNG}
\label{fig:suma}
\end{figure}

\newpage
Zestawienie funkcji $f_p(x)$ oraz $f_n(x)$ wraz ze zmieniającymi się sumami częściowymi szeregu zostały zaprezentowane na dwóch animacjach. Oto jak powstała jedna z animacji:

\smallskip

\begin{figure}[ht!]
\centering
\includegraphics[width=165mm,scale=1]{animacja.PNG}
\label{fig:animacja}
\end{figure}

\bigskip
Gdzie:

\begin{figure}[ht!]
\centering
\includegraphics[width=78mm,scale=1]{gdzie.PNG}
\label{fig:gdzie}
\end{figure}

\end{document}